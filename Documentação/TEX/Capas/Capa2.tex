%%%%%%%%%%%%%%%%%%%%%%%%%%%%%%%%%%%%%%%%%%%%%%%%%%%%%%%%%%%%%%%%%%%%%
%   taisesantiago@gmail.com
%   Modelo para artigos em Portugu�s
%%%%%%%%%%%%%%%%%%%%%%%%%%%%%%%%%%%%%%%%%%%%%%%%%%%%%%%%%%%%%%%%%%%%%

\documentclass[12pt]{article}
\usepackage{geometry}
\usepackage{chngpage}
\usepackage{graphicx}
\usepackage{amsmath}
\usepackage{amsfonts}
\usepackage{amssymb}
\usepackage{latexsym}
\usepackage{indentfirst}
\usepackage[brazil]{varioref}
\usepackage[english,brazil]{babel}
\usepackage[latin1]{inputenc}
\geometry{a4paper,left=2.5cm,right=2.5cm,top=2.5cm,bottom=2.5cm}
\usepackage[dvips]{color}
\usepackage{setspace}


\graphicspath{{figuras/}}
\setstretch{1.5}

%%%%%%%%%%Defini��es Teoremas %%%%%%%%%%%%%%%%%%%

 \newtheorem{teo}{Teorema}%[subsection]
 \newtheorem{cor}[teo]{Corol�rio}
 \newtheorem{lem}[teo]{Lema}
 \newtheorem{prop}[teo]{Proposi��o}
 \newtheorem{defn}[teo]{Defini��o}
 \newtheorem{nota}[teo]{Nota}
 \newtheorem{obs}[teo]{Observa��o}
 \numberwithin{equation}{subsection}
%%%%%%%%%%%%%%%%%%%%%%%%%%%%%

\begin{document}

\thispagestyle{empty}

\title{
	\begin{figure}[!h]
		\centering
    	\includegraphics[scale=1]{unesp3.jpg}
       	\label{figRotulo}
	\end{figure}
	Predi��o de Partidas de Dota2 via Aprendizado de M�quina
	\date{}
	}
\author{Teodoro Balbino Calvo}

\maketitle

\thispagestyle{empty}

\vfill
\hfill
\begin{minipage}{7cm}

Projeto para Trabalho de Conclus�o de Curso apresentado ao Curso de Gradua��o em Estat�stica da FCT/Unesp para aproveitamento na disciplina Trabalho de Conclus�o de Curso.
Orientador: Prof. Dr. Manoel Ivanildo Silvestre Bezerra
	
\end{minipage}


\vfill
\centering
\begin{minipage}[c][4cm][c]{7cm}
	\centering
	Presidente Prudente \\ 2016
\end{minipage}

\pagebreak


\end{document}
